%% LyX 1.6.5 created this file.  For more info, see http://www.lyx.org/.
%% Do not edit unless you really know what you are doing.
\documentclass[oneside,dutch]{amsart}
\usepackage[T1]{fontenc}
\usepackage[latin9]{inputenc}
\usepackage[letterpaper]{geometry}
\geometry{verbose,tmargin=3cm,bmargin=3cm,lmargin=2cm,rmargin=2cm}
\setlength{\parskip}{\smallskipamount}
\setlength{\parindent}{0pt}
\usepackage{amsthm}
\usepackage{graphicx}
\graphicspath{{Figures/}}

\makeatletter
%%%%%%%%%%%%%%%%%%%%%%%%%%%%%% Textclass specific LaTeX commands.
\numberwithin{equation}{section}
\numberwithin{figure}{section}

\makeatother

\usepackage{babel}

\begin{document}

\title{Data verwerking zonder periodieke afhankelijkheid}


\author{N.G. Schultheiss}

\maketitle

\section{Inleiding}

Deze module volgt op de module {}``Data verwerking met periodieke
afhankelijkheid'', waarin werd uitgelegd hoe je een spreadsheet met
gegevens van de detectoren kunt maken. Hier werd uitgelegd hoe conclusies
te trekken zijn, als er sprake is van een periodieke afhankelijkheid.
In het geval van de afhankelijkheid van weersverschijnselen zoals
bliksem, zal dit niet periodiek zijn. Het is namelijk niet zo dat
er om een bepaalde tijd een bliksemflits wordt gegenereerd. 


\section{De opstelling}

Een opstelling die bijvoorbeeld op school gebruikt wordt, bestaat
uit (groepen van) twee detectoren. Om een shower van de achtergrondstraling
te isoleren, zoeken we co�ncidenties. Dit wil zeggen dat er (praktisch)
tegelijk deeltjes in de twee verschillende detectoren worden gedetecteerd.
De shower wordt boven in de atmosfeer gestart door de botsing van
kosmische straling met kerndeeltjes%
\footnote{Verwijzing naar {}``Interactie van kosmische straling en aardatmosfeer''.%
}. Deze botsing veroorzaakt een kettingreactie, waardoor een shower
van deeltjes door de atmosfeer gaat. Een deel van deze shower wordt
in de atmosfeer geabsorbeerd, een deel komt op Aarde bij de detector.
In de praktijk kunnen we zeggen dat de deeltjes bijna loodrecht op
de Aarde vallen, de hoek van inval is meestal kleiner dan $20^{o}$
met het zenith%
\footnote{Het zenith is het punt recht boven je hoofd. Ieder mens (en iedere
plaats) heeft een eigen zenith.%
}.

%
\begin{figure}[h]
\includegraphics[width=17.6cm]{\string"Screenshot-Astronomical Time Calculations\string".png}

\caption{Een conversie pagina}

\end{figure}


Omdat de Aarde ronddraait, draaien de detectoren ook rond. Zoals we
in de module {}``De Hemel'' kunnen zien, is dit draaien van de Aarde
ten opzichte van de Zon of ten opzichte van de sterren te beschouwen.
De gegevens worden aangeleverd met de Aardse tijd. De draaiing van
de Aarde wordt hier dus gegeven ten opzichte van de Zon. Soms willen
we de metingen echter hebben in siderische tijd of ten opzichte van
de sterren. Een conversiepagina is met google te vinden, zoals:

http://www.go.ednet.ns.ca/\textasciitilde{}larry/orbits/jsjdetst.html


\section{Gegevens ophalen}

Gegevens kunnen opgehaald worden op: http://www.hisparc.nl/hisparc-data/data/.
Je kunt deze URL in zijn geheel intypen of naar hisparc.nl gaan. Op
de home-page kun je op het figuurtje van {}``data analyse'' klikken.
Eventueel klik je hierna in de kantlijn op {}``data''. Er verschijnt
een lijst met detectoren. Klik op je op een detector, dan krijg je
een pagina met de gegevens van de detector. Van boven naar beneden
zie je een histogram met het aantal co�ncidenties per uur, een grafiek
met het aantal pulsen tegen de pulshoogte in mV en een grafiek met
het aantal pulsen tegen het oppervlak van de puls in mVps.

Rechts naast de grafieken is een kolom met daarin de datum van de
metingen en verschillende eigenschappen van het meetstation zoals
breedte- (latitude) en lengtegraad (longitude).

Rechts boven de grafieken is {}``Source'' te zien, klik je hierop
dan krijg je een {*}.csv (comma separated values) bestand. Dit is
in een spreadsheet programma te laden. Soms moet je aangeven dat de
informatie door komma's gescheiden is.

%
\begin{figure}[h]
\includegraphics[width=17.6cm]{\string"Screenshot-Data HiSPARC\string".png}

\caption{Het data venster}

\end{figure}


Nu deze gegevens bekend zijn, kunnen we onderzoeken of deze gegevens
afhangen van andere grootheden. 
\end{document}
