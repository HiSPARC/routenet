\documentclass[12pt,a4paper]{article}
\pagestyle{plain}
\usepackage{fullpage}
\usepackage[english]{babel}
\usepackage{enumerate}

%equations
\usepackage[fleqn]{amsmath}
\numberwithin{equation}{section}

%figures
\usepackage[dvips]{graphicx}
\graphicspath{{./images/}}
\numberwithin{figure}{section}

%excercises
\newcounter{Exercise}
\setcounter{Exercise}{1}
\usepackage[dvipsnames]{xcolor}
\usepackage{framed}
\definecolor{shadecolor}{gray}{0.9}
\usepackage{caption}

%tables
\numberwithin{table}{section}

%specials
\usepackage{textcomp} %special (greek) characters as text
%\usepackage{pstricks} %
%\usepackage{ifthen} %
%\usepackage{calc} %
%\usepackage{isotope}
\usepackage{hyperref}
\usepackage[bottom]{footmisc} %footnote below figure
\usepackage{footnpag}%number footnotes per page


%document details
\author{N.G. Schultheiss \\ translated and adapted by K. Schadenberg}
\date{}
\title{Colour}


\begin{document}
\maketitle

\section{Introduction}
What is colour? In the previous module `The Sun' we started to answer this question. In this text we will try to explain what happens when we mix colours and how we can recognize atoms by the light they emit or absorb.

\section{Different colours}
\subsection{Mixing paint}
Using three primary colours, red, yellow, and blue, we can make any colour we desire. Mixing two primary colours in equal amounts gives us a secondary colour; mixing red and yellow gives us orange while blue mixed with yellow gives green, the third and last secondary colour is violet (or purple) obtained by mixing red and blue.

Paint has a certain colour because it absorbs light of different colours. When white light hits a red object we see this object as being red. Isaac Newton explained why this is with his extensive optics experiments. White light consists of all other colours of light: red, orange, yellow, green, blue, violet, and all colours in between, the entire rainbow of colours. When white light hits the red object all light is absorbed except the red light. This is reflected and is what we see when we look at the object.

Newton was not the only one researching light. Johannes Wolfgang Goethe, born after Newtons death and mainly known for writing `Faust' and `Die leiden des jungen Werthers',  was fascinated by the difference in intensity of different colour of light. He used his concept of `Steigerung' (enhancing or increasing) to explain why there are different colours. His reasoning was as follows. When we look at two yellow objects they might not have the same shade of yellow. One may be a very bright yellow such as Signal Yellow\footnote{The RAL colour space system naming is used here.} while the other the more dull yellow colour of egg yolk (Melon Yellow for instance). The Signal Yellow only reflected a part of the Melon Yellow colour and therefore has a different intensity.

Goethe expanded on this idea. According to him there were only two basic colours, yellow and sky blue (cerulean). Blue colours arise when we look through the lightness onto something dark, the way we see the sky on a clear day. Yellow colours on the other hand arise when we look at something bright through the darkness, such as looking at the sun through the dark atmosphere.

It is good to remark that Goethe's `theory' was not a scientific theory as defined by the natural sciences. His description of the nature of colours as published in 'Zur Farbenlehre' (Theory of Colours) was a more philosophical treatment of the subject. It focusses more on the psychological observation of light. As such it was rejected by most physicist but other philosophers used his work to gain new insights in how people perceive colours.

\subsection{Mixing light}
The colour screen of your computer or television does not work with paints absorbing and reflecting different colours of light. Instead it emits different colours of light. Most electronic screens emit three different colours of light; Red, Green, and Blue. The screen are therefore frequently called RGB-screens. When you take a close look at the screens, perhaps with the use of a magnifying glass, you can see the different colour elements.\footnote{A digital image consists of a number of pixels; picture elements. Your screen also has a certain number of pixels. One pixel has one red, one green (sometimes two), and one blue colour element or sub-pixel.}

Mixing of light is different then the mixing of paint. How the mixing of light exactly works you will investigate yourself in the following exercises.

\begin{shaded}
\textbf{Exercise \theExercise \stepcounter{Exercise}} : On the following web page you can combine different amount of red, green, and blue light from a computer screen to obtain different new colours of light: \url{http://www.colortools.net/color_mixer.html}
The amount of light from each colour can be entered in two different ways; as a percentage of the maximum amount, or as a decimal value between 0 and 256. You can also enter a colour number which is a conversion of the decimal values to hexadecimal values.
In which amounts do you need red, green, and blue to make the following colours: yellow, orange, violet, and grey.

Extra: In our decimal (`10') system we use the numbers 0 to 9 to denote a certain value. In the hexadecimal (`16') system we have six more symbols which can be used as numbers, after 8 and 9 we get: A, B, C, D, E, and F. This means that the number 11 in the decimal (base 10) system is equal to B in the hexadecimal (base 16, or hex) system. Convert the following decimal values to their hexadecimal counterpart: 0, 128, and 255.\end{shaded}
\begin{shaded}
\textbf{Exercise \theExercise \stepcounter{Exercise}} : How do you make really intense or dull colours using RGB-coding? Can you make a really bright Signal Yellow?\end{shaded}
\begin{shaded}
\textbf{Exercise \theExercise \stepcounter{Exercise}} : Extra: Why do we see a mixture of red with green light as being orange and not as two separate colours? To answer this question we need to look at how our eye works. Do a little research into the different colour receptors in our eyes and how our brain interprets the signals from our eyes to form a colour picture.\end{shaded}


\section{Table Salt}
In the previous lesson letter from this series (`The Sun') we said that different atoms absorb different colours (wavelengths) of light. In this section we will try and show this with a very common chemical compound. Table salt, sodium chloride (NaCl), is a substance which can be easily bought in high purities. We will try and make sodium yellow with this salt. Sodium yellow is the colour which is emitted by some types of street lamps (sodium-vapour lamp).

For this experiment we will again use a CD. Our light source will be a (small) candle. You can use the CD as a simple mirror to look at the candle, you will see its reflection. If you now tilt the CD a bit in the right direction you will see the spectrum of colours present in the light emitted by the candle. The red colours should be visible near the centre of the disk and the spectrum spreads out towards the edge, here you should see some faint violet colours. Do this experiment in a darkened room to avoid interference from other light sources.

When you can clearly see the spectrum of the candle, drop a small amount of table salt into the flame. You should see the spectrum change, a bright yellow spot appears. To improve the image of the yellow spot you might want to tilt the CD in such a way that the spectrum is oriented horizontally. Then cover up part of the light from the candle so that you end up with a vertical streak of light.

\begin{shaded}
\textbf{Exercise \theExercise \stepcounter{Exercise}} : We can use our CD to investigate the spectrum other light sources. How would you describe the spectrum from the candle? Are there any bright spots or dark regions? How about other light sources? Are there differences between incandescent lights (light bulbs) and fluorescent lamps? \end{shaded}

A simple rule of thumb is that, when a chemical compound absorbs a certain wavelength, it will easily emit that colour of light when the compound is heated. Our sodium salt emitted a bright yellow light when it was heated by the flame from the candle. Does it also absorb this colour? To answer this question we need to build a small experimental setup. Place two candles and a screen in alignment; the first candle in front, then the second candle, and the screen at the end. Position the candles in such a way that the flame from the second candle blocks a part of the light coming from the first candle and thereby casts a shadow on the screen. Now scatter a bit of salt in one of the candles and then in the other. Look at the screen while you sprinkle the salt and see what happens.
\begin{shaded}
\textbf{Exercise \theExercise \stepcounter{Exercise}} : Write down the results from your experiment with two candles. Can you explain the result in your own words?\end{shaded}

\end{document}
